\section*{Введение}

В настоящее время язык \lang{Java} является одним из распространённых языков \cite{tiobe}. Язык \lang{Java} исполняется на платформе \tool{Java Runtime Environment(JRE)}. Благодаря этому программы на нём можно запускать на множестве различных устройств. Для него написано много различных библиотек и инструментов разработки. За время существования языка, выпускалось несколько версий. Одной из ключевых особенностей Java является обратная совместимость языка с предыдущими версиями\cite{openjdk:compatibility}, с другой стороны это тормозит развитие языка и платформы. Во многом поэтому \lang{Java} не может избавиться от громоздких синтаксических конструкций.

В последнее время развиваются языки котрые совместимые с платформой JVM и ранее написанным кодом на \lang{Java}, пытающиеся решить упростить написание программ и решить проблемы языка \lang{Java}. Примерами таких языков являются: \lang{Kotlin}, \lang{Scala}, \lang{Ceylon}, \lang{Clojure}. Каждый из них имеет синтаксические конструкции, упрощающие разработку. Одной из таких конструкций являются протоколы: типы со структурным отношением подтипизации.

\lang{Kotlin} язык совместимый с кодом написанным на \lang{Java}. На данный момент его основной платформой является виртуальная машина \lang{Java}(\tool{JVM}). В данной работе рассматриваются особенности реализации типов протоколов в \tool{JVM}, описана реализация протоколов для языка \lang{Kotlin}, приведены сравнения реализации с другими языками в аспектах производительности и выразительности.
