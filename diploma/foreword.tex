\section*{Введение}

В большинстве современных языков программирования существует механизм наследования \cite{Cook:1989:IS:96709.96721}. Он вводит над типами отношение подтипов явно, с помощью средств языка. Однако, ручное указание подтипизации вводит ограничение на переиспользование типов из сторонних библиотек.

В качестве альтернативного подхода существует структурная типизация\cite{book:pierce}. В языках с такой типизацией отношение подтипов вычисляет компилятор. Такой подход менее распространён из-за сложности его использования. Однако существует промежуточное решение - интеграция структурных типов в языки с номинальной типизацией. Такие типы называются протоколами.

В настоящее время язык программирования \lang{Java} является одним из распространённых \cite{tiobe}. Он исполняется на платформе \tool{Java} \tool{Runtime} \tool{Environment(JRE)}. Благодаря этому, программы на нём можно запускать на множестве различных устройств. Для платформы написано большое количество библиотек и инструментов разработки. За время существования языка выпускалось несколько его версий.

Одним из ключевых достоинств \lang{Java} является обратная совместимость языка с предыдущими версиями\cite{openjdk:compatibility}. С одной стороны, это позволяет использовать части приложений, написанных для разных версий, с другой, тормозит развитие языка и платформы. Вследствие чего развиваются языки, которые совместимы с платформой \tool{JVM} и ранее написанным кодом на \lang{Java}, пытающиеся упростить написание программ и решить проблемы языка \lang{Java}. Примерами таких языков являются: \lang{Kotlin}, \lang{Scala}, \lang{Ceylon}, \lang{Clojure}. Каждый из них имеет синтаксические конструкции, упрощающие разработку.

\lang{Kotlin} - это язык программирования совместимый с кодом, написанным на \lang{Java}. Его основной платформой является \tool{JVM}. Как и в \lang{Java}, в языке \lang{Kotlin} поддерживается отношение наследования. В данной работе рассматриваются особенности реализации протоколов для \tool{JVM} в контексте языка программирования \lang{Kotlin}.
