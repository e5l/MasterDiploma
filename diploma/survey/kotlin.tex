\subsection{Язык Kotlin}

Целевой платфомой языка Kotlin является JVM. Kotlin поддерживает совместимость с языком Java\cite{kotlin:compatibility}, поддерживается раздельная компиляция. Не смотря на это существуют большие различия.

В Kotlin присутствует собственный набор встроенных типов: Int, Short, Boolean, и т.д. Данные типы не могут хранить пустую ссылку. Для объявления типов, в которых можно хранить пустую ссылку существует специальный синтаксис: Int?, Short?, Boolean?. Благодаря тому что типы проаннотированы, в процессе компиляции Kotlin проверяет корректность присваивания на уровне типов. Во время трансляции встроенный тип может стать как примитивным так и ссылочным, взависимости от контекста использования и возможности хранения пустой ссылки. Например Int? будет оттранслирован в Integer.

Как и Java, Kotlin использует номинальную систему типов с наследованием. Поддерживается механизм интерфейсов. В языке присутствуют операторы is и as. С помощью первого во время выполнения можно проверить является ли объект экземпляром указанного типа. Второй оператор пытается выполнить приведение типов.

Одной из больших отличий Kotlin от Java является наличие смешанной вариантности в типах шаблонов. Kotlin позволяет задать отношение подтипизации на параметрических типах, как в месте использования(как в Java), так и в месте объявления(как в Scala) типа.
