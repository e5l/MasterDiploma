\subsection{Язык Kotlin}

Целевой платфомой языка Kotlin является JVM. Kotlin поддерживает совместимость с языком Java\cite{kotlin:compatibility}, поддерживается раздельная компиляция. Не смотря на это существуют большие различия.

Для абстрактных методов в интерфейсах и класах в Kotlin можно задавать реализацию по умолчанию. Для методов присутствует перегрузка по параметрам. Аргументы могут иметь значения по умолчанию.

\subsubsection{Типизация}
В Kotlin присутствует проверка пустоты ссылок на уровне системы типов. В обычные ссылки не возможно положить пустую. Для работы с пустыми ссылками существует специальный синтаксис.

В Kotlin присутствует собственный набор встроенных типов: Int, Short, Boolean, и т.д. Для объявления типов, в которых можно хранить пустую ссылку в конце добавляется вопросительный знак: Int?, Short?, Boolean?. Благодаря тому что типы проаннотированы, в процессе компиляции Kotlin проверяет корректность присваивания на уровне типов. Во время трансляции встроенный тип может стать как примитивным так и ссылочным, взависимости от контекста использования и возможности хранения пустой ссылки. Например в большинстве случаев Int? будет оттранслирован в Integer, а Int в int.

Kotlin использует номинальную систему типов с наследованием. Поддерживается механизм интерфейсов. В языке присутствуют операторы is и as. С помощью первого во время выполнения можно проверить является ли объект экземпляром указанного типа. Второй оператор пытается выполнить приведение типов.

В Kotlin существует возможность создавать интерфейсы. В интерфейсах можно задать реализацию методов по умолчанию.

\paragraph{Параметрические типы}
В Kotlin существует возможность создавать параметрические типы. Тип параметра указывается в треугольных скобках. На параметрических типах можно задавать отношение порядка в зависимости от типа параметра. Это можно сделать как и в объявлении типа так и при определении шаблонного метода.
