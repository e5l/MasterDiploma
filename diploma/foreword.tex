\section*{Введение}

В настоящее время язык Java является одним из распространённых языков \cite{tiobe}. Язык Java исполняется на платформе Java Runtime Environment(JRE). Благодаря этому программы на нём можно запускать на множестве различных устройств. Для него написано много различных библиотек и инструментов разработки. За время существования языка, выпускалось несколько версий. Одной из ключевых особенностей Java является обратная совместимость языка с предыдущими версиями\cite{openjdk:compatibility}, с другой стороны это тормозит развитие языка и платформы. Во многом поэтому Java не может избавиться от громоздких синтаксических конструкций.

В последнее время развиваются языки котрые совместимые с платформой JVM и ранее написанным кодом на Java, пытающиеся решить упростить написание программ и решить проблемы языка Java. Примерами таких языков являются: Kotlin, Scala, Ceylon, Clojure. Каждый из них имеет синтаксические конструкции, упрощающие разработку. Одной из таких конструкций являются протоколы: типы со структурным отношением подтипизации.

Kotlin язык совместимый с кодом написанным на Java. На данный момент его основной платформой является виртуальная машина Java(JVM). В данной работе рассматриваются особенности реализации типов протоколов в JVM, описана реализация протоколов для языка Kotlin, приведены сравнения реализации с другими языками в аспектах производительности и выразительности.
