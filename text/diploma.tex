\documentclass{spbau-diploma}
\begin{document}
% Год, город, название университета и факультета предопределены,
% но можно и поменять.
% Если англоязычная титульная страница не нужна, то ее можно просто удалить.
\filltitle{ru}{
    chair              = {Кафедра математических и информационных технологий},
    title              = {Протоколы/Структурные типы в Kotlin},
    % Здесь указывается тип работы. Возможные значения:
    %   coursework - Курсовая работа
    %   diploma - Диплом специалиста
    %   master - Диплом магистра
    %   bachelor - Диплом бакалавра
    type               = {master},
    position           = {студента},
    group              = SE,
    author             = {Сташевский Леонид Евгеньевич},
    supervisorPosition = {д.\,ф.-м.\,н., профессор},
    supervisor         = {Жарков Д.\,Буква.},
    reviewerPosition   = {ст. преп.},
    reviewer           = {Привалов А.\,И.},
    chairHeadPosition  = {д.\,ф.-м.\,н., профессор},
    chairHead          = {Омельченко А.\,В.},
    % university = {САНКТ-ПЕТЕРБУРГСКИЙ АКАДЕМИЧЕСКИЙ УНИВЕРСИТЕТ},
    % faculty = {Центр высшего образования},
    % city = {Санкт-Петербург},
    % year             = {2013}
}
\maketitle
\tableofcontents
% У введения нет номера главы
\section*{Введение}
В современных языках программирования существует много способов вводить отношение на типах. Самый популярный подход - отношение наследования или номинальная типизация, где пользователь сам решает является ли один объект подтипом другого. Существует вариант автоматического вывода отношения подтипизации на основе интерфейса объектов. Такой способ называется структурным. Номинальная типизация считается сильно проще структурной и присутствует практически во всех языках программирования со статической типизацией(примеры). Структурная считается сложней, но является более выразительной, поэтому используется в функциональных языках(примеры). Протокол это структурный тип в номинальном языке.

На данный момент набирает популярность язык Kotlin. Kotlin использует номинальную систему типов и его основной целевой платформой является виртуальная машина языка Java.
% Обоснование выбора языка Kotlin
%Язык Kotlin популярен на сегодняшний день. Во многих языках поддерживается структурная типизация.
% Абзац что данная работа про Kotlin а значит и про JVM
% Абзац что под JVM были попытки сделать протоколы


В данной работе описано:
\begin{itemize}
    \item Исследование способов реализации структурных типов в JVM, существующие реализации и выяснение ограничений
    \item Прототипа и тестирование производительности
    \item Реализация проверки типов в компиляторе языка Kotlin
    \item Генерация кода для компилятора
    \item Оптимизации и поддержка многопоточности
\end{itemize}
\section{План}

\begin{itemize}
    \item введение в типизацию
    \item постановка задачи: структурная типизация
    \item почему и кому это нужно: примеры использования, востребованность в других языках
    \item обзор реализаций и статей: scala, whiteoak, go. посмотреть принятые решения там
    \item Реализация: бэкэнд, фронтенд
    \item Фронтенд: рассказать какие случаи в языке допустимы для проверки типов, привести крайние случаи и примеры
    \item Бэкэнд: варианты реализации возможные, что используется в других языках на бэкэнде. Выбор между гибкостью и скоростью. Прототип и тесты производительности. В каких местах и что быстрей. Кэширование: как в других языках, в чём смысл. Сколько в процентном соотношении занимает разрешение ссылки на метод.
    \item Имплементация которая выбрана и реализована(или несколько). (?Детали реализации в компиляторе? какие? стоит ли?)
    \item Табличка с результатами сравнения
    \item Заключение
\end{itemize}

% У заключения нет номера главы
\section*{Заключение}

\bibliographystyle{ugost2008ls}
\bibliography{diploma.bib}
\end{document}
