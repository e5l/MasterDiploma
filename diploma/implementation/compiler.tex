\subsection{Реализация протоколов в компиляторе Kotlin}
В данном разделе будет рассказано о деталях реализации протоколов в компиляторе kotlin.

\subsubsection{Функциональность}

\paragraph{Разрешение перегрузок}
Рассмотрим проблему несовпадения типов при использовании примитивных типов. Поиск методов выполняется по точному типу функции. Мы проверяем что необходимый метод существует во время компиляции. Проблема возникает в том что при трансляции конечные типы не совпадают. Данную проблему можно решить во время выполнения. Для этого необходимо изменить поиск метода из библиотеки на алгоритм разрешения перегрузок. Идея заключается в том, чтоб выбирать ближайшую по расстоянию функцию.

Для выбора перегрузки, для каждого метода в классе с таким же именем считается вектор расстояния. Каждая компонента этого вектора - расстояние между типами аргументов в типе метода и типе сохранённом во время компиляции если они в одной иерархии наследования, в противном случаее -1. Из всех векторов выбирается тот, у которого все координаты не меньше чем у любого другого вектора. Наличие и единственность такого метода будут гарантированы во время компиляции при приведении типов.

\paragraph{Наследование и наличие типа}
Ещё одной из проблем при полном стирании типа является невозможность наследования: при наследовании необходимо написать тип родителя. Одним из очевидных решений является частичное стирание типа: для генерации интерфейса тип и при указании типов базового класса вручную тип сохраняется. В типах методов и объектах тип по прежнему стирается. Таким образом возможно создание анонимного оъекта, который реализует протокол.

\subsubsection{Синтаксис}
Для поддержки протоколов в компиляторе был введён новый тип интерфейсов и добавлено ключевое слово protocol.

\begin{pyglist}[language=kotlin]
protocol interface Proto {
    fun id(i: Int): Int
}

class Impl {
    fun id(i: Int): Int = i
}

fun foo(arg: Proto) {
    println(arg.id(42))
}
foo(Impl())
\end{pyglist}

% проверка типов - проверка возможности сделать оверрайд
% генерация интерфейса => анонимные имплементации
% райнтайм - выбор перегрузок
% кэш

