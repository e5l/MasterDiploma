\subsubsection{Scala}
Рассмотрим реализацию протоколов в языке Scala\cite{scala:structural}. Scala компилируется в JVM байткод. Реализацию можно разделить на 2 части: проверку типов при компиляции и генерация байткода для вызова метода.

Структурные типы являются обобщением типов Scala. Проверка структурных типов изначально поддерживается компилятором с одним исключением: накладывается ограничение на явное наследование. Поэтому для проверки структурных типов достаточно отключить только одну проверку.

Рассмотрим генерацию байткода. В Scala типы протоколов существуют только во время компиляции, в байткоде не генерируется новый тип. В процессе компиляции, тип протокола стираются до типа Object. Вызов метода у объекта стркутурного типа происходит с помощью специального механизма applyDynamic: для каждого места обращения к полю объекта генерируется статический метод. В статическом методе происходит поиск нужного метода. Сперва метод ищется в кэше, затем с помощью рефлексии. Так как внутри вызова может произойти исключительная ситуация, в этом методе происходит обработка исключения и разворачивание исходного исключения из InvokationTargetException. Таким образом вызов дополнительного метода выглядит прозрачно для пользователя. В реализации используют кэширование для всех найденных методов. В качестве кэша используют ссылочный список, где сохраняют каждый увиденный объект. В связи с тем что во время выполнения отсутствуют типы - запрещено создание generic протоколов или использование в протоколах generic типов.

Внимательное тестирование подхода, используемого в Scala показало что есть возможность написать код, порождающий некоректное поведение и ошибку времени выполнения, несмотря на внешнюю корректность синтаксиса. Такое поведение связано с использованием примитивных типов в generic классах:

\begin{minted}[scala]
class Impl[T] {
	def x(i: T): T = i
}

class Main {
	type Proto = { def x(i: Int): Int }

	def foo(arg: Proto) {
		print(arg.x(42))
	}

	def main(args: Array[String]) {
		foo(Impl[Int]())
	}
}
\end{minted}

С точки зрения языка Scala такой код является корректным. Он проходит проверку типов и компиляцию, но во время выполнения происходит искючительная ситуация. Причиной этому является не совпадение сигнатур методов. Во время генерации класса Impl тип T становится наиболее общим типом Object. В то же время для вызова в методе foo сгенерировался метод ищущий функции с типом Int. Поэтому в во время выполнения нужная функция в объекте Impl отсутствует и происходит исключение.
