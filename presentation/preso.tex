%% Простая презентация с примером включения программного кода и
%% пошаговых спецэффектов
\documentclass{beamer}
\usetheme{SPbAU}
%\useoutertheme{infolines}
\usepackage{fontspec}
\usepackage{xunicode}
\usepackage{xltxtra}
\usepackage{xecyr}
\usepackage{hyperref}
\setmainfont[Mapping=tex-text]{DejaVu Serif}
\setsansfont[Mapping=tex-text]{DejaVu Sans}
\setmonofont[Mapping=tex-text]{DejaVu Sans Mono}
\usepackage{polyglossia}
\setdefaultlanguage{russian}
\usepackage{graphicx}
\usepackage{listings}
\lstdefinestyle{mycode}{
  belowcaptionskip=1\baselineskip,
  breaklines=true,
  xleftmargin=\parindent,
  showstringspaces=false,
  basicstyle=\footnotesize\ttfamily,
  keywordstyle=\bfseries,
  commentstyle=\itshape\color{gray!40!black},
  stringstyle=\color{red},
  numbers=left,
  numbersep=5pt,
  numberstyle=\tiny\color{gray},
}
\lstset{escapechar=@,style=mycode}

\begin{document}
\title[Структурный типы в языке Kotlin]{Реализация структурных типов в языке Kotlin}
% \subtitle{предварительные результаты}
\author[Сташевский Л.Е.]{Сташевский Леонид Евгеньевич\\{\footnotesize\textcolor{gray}{научный руководитель: Д.С. Жарков}}}
\institute{СПб АУ НОЦНТ РАН}
\frame{\titlepage}

\begin{frame}\frametitle{Введение}
Протокол - структурный тип в языке с номинальной типизацией.
\end{frame}

\begin{frame}\frametitle{Цель и задачи}
Цель данной работы реализация протоколов в языке Kotlin

Задачи 
\begin{itemize}
    \item Изучить существующие подходы реализации протоколов
    \item Оценить возможности ограничения платформы
    \item Реализовать поддержку протоколов в компиляторе Kotlin
    \item Измерить время вызова методов
\end{itemize}
\end{frame}


\begin{frame}\frametitle{Подходы к реализации}
Способы реализации
\begin{itemize}
    \item Библиотека рефлексии
    \item Библиотека вызовов
    \item Генерация кода для вызова методов
\end{itemize}
Рассмотренные реализации
\begin{itemize}
    \item Scala
    \item Whiteoak
    \item Go
\end{itemize}
\end{frame}

\begin{frame}\frametitle{Найденные проблемы}
\begin{itemize}
    \item Методы с примитивными типами
    \item Перегрузки
    \item Проблема с областью видимости
    \item Шаблонные типы
    \item Производтильетность
\end{itemize}
Другие подходы к реализации порождают проблему потери идентичности объектов.
\end{frame}


\begin{frame}\frametitle{Прототип}
Для сравнения производительности реализаций было принято решение написать прототип.
% таблица с производительностью
\begin{table}[h]
\begin{tabular}{|c|c|c|} \hline
Бенчмарк & Рефлексия (нс) & Invoke(нс) \\ \hline
Класс & 4.815 $\pm$ 0.072 & 5.002 $\pm$ 0.031 \\ \hline
Наследник & 4.797 $\pm$ 0.023 & 5.002 $\pm$ 0.037 \\ \hline
Интерфейс & 4.818 $\pm$ 0.055 & 5.016 $\pm$ 0.104 \\ \hline
Анонимная функция & 153.892 $\pm$ 0.909 & 5.069 $\pm$ 0.111 \\ \hline
\end{tabular}
\end{table}
\end{frame}

\lstset{language=kotlin}
\begin{frame}[fragile]\frametitle{Компилятор: синтаксис}
\begin{lstlisting}
protocol interface Foo {
    fun id(i: Int): String
}

class X {
    fun id(i: Int) = i.toString()
}

...

val y: Foo = X()
\end{lstlisting}
\end{frame}

\begin{frame}\frametitle{Компилятор: детали реализации}
\begin{itemize}
    \item
\end{itemize}
\end{frame}

\begin{frame}\frametitle{Компилятор: возможности}
\end{frame}

\begin{frame}\frametitle{Компилятор: измерения}
\end{frame}

\begin{frame}\frametitle{Результаты}
\begin{itemize}
    \item Реализована поддержка протоколов в языке Kotlin
    \item Найдены ошибки в реализации протоколов в других языках
    \item Измерены различные подходы в языке
\end{itemize}
\end{frame}
\end{document}