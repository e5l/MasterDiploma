\section*{Введение}
В современных языках программирования существует много способов вводить отношение подтипизации. Самый популярный подход - это отношение наследования или номинальная типизация. В наследовании программист сам решает какой объект является наследником другого. На языке типов это означает что тип наследника является подтипом типа предка. Существует вариант автоматического вывода отношения подтипизации на основе интерфейса объектов. В процессе сборки, компилятор проверяет что используемый тип удовлетворяет ограничениям указанного типа. Такая система типов называется структурной. Такая типизация встречается в некоторых языках программирования: Go, семейство ML.

Номинальная типизация является простой, а следовательно и более распространённой. Она используется в таких языках как Java, C++, Kotlin. Вместе с простотой на программиста накладываются ограничения: при взаимодействии с кодом, который уже скомпилирован в библиотеку. Типы находящиеся в других модулях нельзя добавить в иерархию наследования, поэтому для них приходится писать и поддерживать типы-обёртки.

Существует другое решение, сохраняющее простоту и выразительность. Можно использовать номинальную типизацию в большинстве случаев, а там где необходима гибкость - использовать структурный тип. Такой подход сохранит совместимость с уже написанным кодом и позволит писать более выразительный код. В языке с номинальными типами структурные типы называют протоколами.

На данный момент набирает популярность язык Kotlin. В нём используется номинальная система типов. На момент написания его основной целевой платформой является виртуальная машина языка Java. Существуют несколько языков на основе JVM, в которых поддерживаются структурные типы. Например они поддерживаются языками Scala, Whiteoak.
