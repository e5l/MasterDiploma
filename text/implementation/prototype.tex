\subsection{Прототип}

Детальнее рассмотрим требования, которые должна соблюдать реализация.

\begin{itemize}
\item Язык Kotlin совместим с Java. Это значит, что объекты из языка Kotlin должны быть корректными объектами для языка Java и в обратную сторону. Одной из важных особенностей, которую хочется сохранить - это идентичность объекта, то есть приведение к структурному типу должно не модифицировать ссылку на объект.
\item Другим важным требованием является возможность использования структурных типов как в качестве типов коллекций, так и в качестве элементов существующих коллекций. Таким образом необходимо, чтоб структурные типы поддерживали типовые параметры.
\item Протоколы не должны вносить ощутимых расходов на потребление памяти и время работы при приведении типов и вызове метода.
\item Возможности использование структурных типов должны быть схожи с обычными типами.
\item По возможности необходимо избегать генерации кода в процессе выполнения
\end{itemize}

Вариант в котором структурный тип оборачивается в класс обёртку не подходит. При таком подходе теряется равеноство ссылок между исходным объектом и обёрнутым. Поэтому необходимо рассматривать подходы, при которых используются исходные ссылки на объект.

Переформулируя получаем что для реализации структурных типов необходимо уметь производить поиск и вызов любой функции на произвольном объекте. Такой функциональности можно добиться используя одну из двух стандартных библеотек: библиотеку рефлексии или библиотеку invoke. Поскольку обе библиотеки предоставляют схожую функциональность, выбор между ними сводится к поиску более производительного решения. Для того чтоб выбрать между ними реализован прототип, который осуществляет генерацию байткода для двух подоходов