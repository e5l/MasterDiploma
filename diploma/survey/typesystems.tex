\subsection{Структурная и номинальная типизации}
В данном разделе рассмотрены основы структурной типизации и номинальной типизации.

\paragraph{Номинальная типизаци}
Номинальная типизация во многих языках реализованна с помощью механизма наследования. Пользователь явно указывает наследников каждого класса. Исходя из этого, компилятор разрешает вызовы методов. В некоторых языках происходят опциональные проверки перегрузок.

У номинальной типизации существует множество плюсов. В первую очередь она является интуитивной для программиста: программист знает какие классы могут быть переданы и явно поддерживает иерархию наследования. В процессе компиляции можно использовать информацию о наследовании и расположить поля объектов так, чтоб к ним удобно было обращаться по смешению, тем самым существенно уменьшается время доступа к полям объектов.

Поведя итог можно привести следующие плюсы номинальной типизации:
\begin{itemize}
    \item быстрее во время выполнения
    \item помогает избежать случайных отношений
    \item проще объявлять рекурсивные типы
    \item быстрее проверять
\end{itemize}

\paragraph{Структурная типизация}
Структурная типизация менее распространена. Очень часто она используется в академических целях и в основном поддерживается в функциональных языках программирования. Структурная типизация схожа с динамической за исключением проверки наличия методов и полей на этапе компиляции.

Тип $A$ является подтипом типа $B$ если выполнены следующие условия:
\begin{itemize}
    \item $A$ содержит все именованные поля $B$
    \item Тип каждого поля $A$ является подтипом $B$
\end{itemize}
Любой тип является подтипом пустого типа. Типы функций являются подтипами только при полном совпадении типов аргументов и возвращаемого значения.

У структурной типизации есть свои плюсы. Она является более гибкой: имена типов не имеют значения, имеет значение только содержимое. Имена типов при этом не имеют значения: два типа с одинаковым содержимым. Проверка типов при структурной типизации: проверка вложенности двух типов.

Плюсы структурной типизации:
\begin{itemize}
    \item замкнуты: тип содержит в себе всё описание
    \item имя типа имеет символический характер и не влияет на отношение типизации
\end{itemize}