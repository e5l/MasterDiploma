\section*{Заключение}
Данная работа посвящена протоколам в языке программирования \lang{Kotlin}. В работе были рассмотрены способы реализации протоколов для различных платформ. Была предложена реализация протоколов в языке \lang{Kotlin} для платформы \tool{JVM}. Приводится качественное сравнение полученной реализации с реализациями в других языках. В отличии от других языков поддерживается раздельная компиляция и сохраняется идентичность объектов, приведённых к типу протокола; реализована поддержка работы с параметрическими и примитивными типами, перегрузки методов по аргументам.

Для работы не требуется генерация новых типов во время выполнения, вместо этого, для вызова метода используются библиотека рефлексии и библиотека вызовов. В отличии от реализаций в других языках используется алгоритмы разрешения перегрузок во время выполнения. Для уменьшения расходов на поиск метода при вызове, было использовано кеширование.

По ходу реализации был написан прототип и набор бенчмарков, для сравнения полученной реализации с реализациями в других языках. В общем случае решение работает не медленее решения в языке \lang{Scala}. Было выявлено, что при использовании библиотеки вызовов время вызова метода протокола не зависит от числа аргументов. В случае вызова методов с более чем одним аргументом, полученная реализация существенно превосходит реализацию в языке \term{Scala}.

Исходные коды полученой реализации, прототипа и бенчмарков, а так же результаты измерений доступны в репозитории\cite{repo}.
