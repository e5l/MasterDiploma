
\subsection{Постановка проблемы}

Рассмотрим обычную ситуацию использования протокола с точки зрения пользователя. Существует протокол, являющийся структурным типом. Существует уже заранее скомпилированный тип, который содержит методы протокола(а значит и является его подтипом), но явное наследование не указывается. С точки зрения виртуальной машины, привести существующий тип можно лишь к тому типу, который содержится в его иерархии наследования. Протокол был создан независимо от подходящего типа, поэтому в иерархии наследования он отсутствует. Более того, в месте вызова гипотетически может быть любой тип, удовлетворяющий требованиям протокола. Это значит, что тип во время компиляции не известен и приведение типа к типу протокола невозможно. Следовательно, невозможно сгенерировать вызов с помощью любой инструкции в момент компиляции.

Для решения данной проблемы существует два подхода. Первый подход заключается в генерации класса обёртки для каждого типа, который приводится к структурной и делегирует вызов объекту конкретного типа. В момент генерации класса обёртки должен быть извесен тип объекта, который приводится к протоколу, а значит, либо необходима информация о всех типах в момент компиляции, либо обёртки для примитивных типов придётся генерировать в момент первого приведения каждого типа. Таким образом, необходимо, либо генерировать новые классы во время выполнения, либо отказаться от раздельной компиляции. Более того, возникает проблема определения идентичности объекта: дважды приведённый объект будет иметь разные экземпляры классов обёрток и проверка на идентичность не пройдёт.

Второй подход заключается в использовании библиотек, которые работают с информацией об объекте во время выполнения: библиотеки вызовов или библиотеки invoke. При таком подходе, информацию о типах знать не нужно, её можно получить во время компиляции. Данный подход может работать медленнее, но не накладывает ограничений на разработку компилятора, поэтому он был выбран в качестве базового подхода.

\subsection{Прототип}
Для реализации структурных типов необходимо уметь производить поиск и вызов любой функции на произвольном объекте. Такой функциональности можно добиться, используя одну из двух стандартных библиотек: библиотеку рефлексии или библиотеку вызовов. Поскольку в данном контексте обе библиотеки предоставляют схожую функциональность, выбор между ними сводится к поиску более производительного решения. Для того, чтобы выбрать между ними, реализован прототип, задачей которого является измерение производительности двух подходов. В качестве модели была выбрана ситуация вызова одного метода протокола с одним ссылочным аргументом, возвращающая ссылочный аргумент. В прототипе осуществляется ручная генерация байткода для двух подоходов.

\subsubsection{Реализация с помощью библиотеки рефлексии}
Рассмотрим реализацию с помощью библиотеки рефлексии. Для места вызова метода протокола генерируется статический метод. Этот метод принимает на вход объект, на котором происходит вызов, и возращает экземпляр класса \method{Method}. После этого, все аргументы упаковываются в массив и вызывается метод \method{invoke}, который делает вызов нужного метода и сохраняет на стек результат. Поскольку сигнатура метода \method{invoke} не типизирована, то необходимо выполнить приведение типов. Внутри статического метода делается обращение к классу на поиск метода с нужным именем и типом. При использовании протколов имеет место следующее предположение: типов, которые будут приведены к протколу в конкретном месте вызова, будет значительно меньше, чем самих вызовов метода. Тип и имя метода известны в момент компиляции и не меняются в процессе выполнения. Процедура поиска метода происходит значительно дольше чем вызов, поэтому для поиска метода выгодно использовать кеширование результата. Так как тип объекта на котором осуществляется вызов один, результат поиска сохраняется в статическое поле класса при первом вызове и используется как результат в дальнейшем.

\subsubsection{Реализация с помощью библиотеки вызовов}
Рассмотрим реализацию с помощью библиотеки вызовов. Аналогично предыдущему варианту, для каждого места вызова генерируется статический метод. В этом вызове происходит вызов инструкции \method{invokedynamic}, которой передаётся ссылка на вспомогательный статический метод и тип вызываемого метода. Помимо этого, \tool{JVM} передаёт экземпляр класса \method{Lookup} для данного места вызова. Внутри вспомогательного метода создаётся объект типа \method{ConstantCallSite}, который содержит в себе вспомогательный класс с функциями отвечающими за поиск нужного метода. Для поиска метода происходит вызов метода \method{find} для объекта, который возвращает экземпляр класса \method{MethodHandle}, ссылающийся на необходимый метод. Из вызова статического метода возвращается \method{MethodHandle}, вызов у него \method{invokeExact}, вызывает необходимый метод. В отличии от механизма рефлексии, оборачивание аргументов в массив не требуется. Помимо \method{invokeExact} существует так же метод \method{invoke}, который в случае необходимости производит приведение типов.

\subsubsection{Сравнение производительности}
Цель данного измерения - понять скорость вызова метода с помощью двух решений в различных случаях. В ходе различных тестирований была выявлена зависимость между типом реализации вызываемого объекта и временем работы, данный критерий был взят в качестве параметра тестирования. Для тестирования были выделены следующие типы целевых объектов
\begin{itemize}
  \item финальный метод класса;
  \item переопределённый метод наследника, вызванный через ссылку на базовый класс;
  \item метод интерфейса, реализовынный классом;
  \item метод интерфейса, реализованный анонимной функцией.
\end{itemize}

Тестирование производилось для функции с одним аргументом, внутри которой вызывается специальный метод JMH предотвращающий удаление непродуктивного кода. Вызовы производились в одном потоке. В процессе тестирования было обнаружено, что существует влияние типа реализации протокола на скорость вызова. Время выполнения измерялось в наносекундах на вызов, каждый бенчмарк запускался в течении секунды, результаты были усреднены по 20 запускам. Результаты измерений для рефлексии приведены в таблице \ref{benchmark:prototype}.
\begin{table}
\begin{center}
\begin{tabular}{|c|c|c|c|} \hline
Тип метода & Рефлексия(нс) &  Вызовы(нс) & Фактор \\ \hline
Класс & 4.815 $\pm$ 0.072 & 5.002 $\pm$ 0.031 & 0.963 \\ \hline
Наследник & 4.797 $\pm$ 0.023 & 5.002 $\pm$ 0.037 & 0.959 \\ \hline
Интерфейс & 4.818 $\pm$ 0.055 & 5.016 $\pm$ 0.104 & 0.961 \\ \hline
Анонимная функция & 153.892 $\pm$ 0.909 & 5.069 $\pm$ 0.111 & 30.359 \\ \hline
\end{tabular}
\caption{Результаты бенчмарка для реализации с помощью рефлексии}
\label{benchmark:prototype}
\end{center}
\end{table}

Из результатов измерения видно, что в среднем библиотека вызовов немного медленнее чем библиотека рефлексии. Однако, существуют ситуации, в которых библиотека рефлексии в 30 раз медленее. Для поиска причин замедления было принято решение изучить машинные инструкции, с целью найти причину замедления и, по возможности, устранить её. Для этого были сделаны снимки с помощью утилиты \tool{perfasm}, которая позволяет вывести горячие точки - байткод и машинные инструкции, выполнявшиеся дольше всего. Листинг горячих точек можно найти в приложении 1. Изучение горячих точек показало, что основное время выполнения происходит внутри исходного кода виртуальной машины, в пределах одной инструкции вызова. Таким образом, оптимизации, с точки зрения байткода, не представляются возможными.

После оценки результатов измерений было принято решение реализовать оба способа в компиляторе и предоставить пользователю возможность выбрать одну из реализаций с помощью опций компиляции. Обе библиотеки достаточно динамично развиваются, поэтому в будующем возможна оптимизация отдельных случаев использования в виртуальной машине.
