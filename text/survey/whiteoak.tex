\subsubsection{Whiteoak}
Whiteoak - это модификация языка Java с целью добавить поддержку структурных типов. Whiteoak модифицирует компилятор Java: добавляет новую синтаксическую конструкцию и её поведение в байткоде. Для обеспечения функциональности происходит генерация класса обёртки для каждого протокола. Каждый объект заворачивается в такую обёртку. Обёртка содержит в себе вызовы, которые перенаправляют вызов реальному объекту. Так как тип объекта для вызова не извесен во время компиляции, Whiteoak генерирует класс обёртку во время выполнения, для конкретного типа.

Преимущество данного подхода - высокая скорость работы. После генерации класса обёртки, скорость вызова метода протокола сводится к двум выполнениям инструкции invokevirtual, что практически не отличимо от обычного вызова метода.

Существенным недостатком данного подхода является потеря идентичности объекта при оборачивании. Дело в том что один и тот же объект, дважды приведённый к типу протокола, имеет разные обёртки, а следовательно и разные ссылки. В Whiteoak проблема решена следующим образом: везде где пользователь использует объект как сущность, используется ссылка на реальный объект. Это вносит определённые ограничения на использование протоколов в типовых параметрах. Появляется неоднозначность, какой объект ожидает метод: обёртку или ссылку. Эта проблема не решается в момент выполнения, поэтому Whiteoak запрещает использование параметрических типов для протоколов.
