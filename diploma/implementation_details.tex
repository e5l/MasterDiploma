\section{Детали реализации}
В данном разделе рассмотрены и обоснованы выбранные решения, приведены сравнения производительности и выразительности языков.

\subsection{Требования к реализации}
Перед тем как приступить к реализации необходимо определить какие ограничения необходимо соблюдать для получения реализации которую удобно использовать на практике

\begin{itemize}
\item Язык Kotlin совместим с Java. Это значит, что объекты из языка Kotlin должны быть корректными объектами для языка Java и в обратную сторону. Одной из важных особенностей, которую хочется сохранить - это идентичность объекта, то есть приведение к структурному типу должно не модифицировать ссылку на объект.
\item Другим важным требованием является возможность использования структурных типов как в качестве типов коллекций, так и в качестве элементов существующих коллекций. Таким образом необходимо, чтоб структурные типы поддерживали типовые параметры.
\item Протоколы не должны вносить ощутимых расходов на потребление памяти и время работы при приведении типов и вызове метода.
\item Возможности использование структурных типов должны быть схожи с обычными типами.
\item Необходимо избегать генерации кода в процессе выполнения
\end{itemize}

Данные ограничения в основном влияют на сгенерированный код. Поэтому сперва рассмотрены подходы для генерации и функциональность которая может быть получена, а затем возможности особенности реализации в компиляторе.


\subsection{Прототип}
Вариант в котором структурный тип оборачивается в класс обёртку не подходит. При таком подходе теряется равеноство ссылок между исходным объектом и обёрнутым. Поэтому необходимо рассматривать подходы, при которых используются исходные ссылки на объект.

Переформулируя получаем что для реализации структурных типов необходимо уметь производить поиск и вызов любой функции на произвольном объекте. Такой функциональности можно добиться используя одну из двух стандартных библеотек: библиотеку рефлексии или библиотеку invoke. Поскольку обе библиотеки предоставляют схожую функциональность, выбор между ними сводится к поиску более производительного решения. Для того чтоб выбрать между ними реализован прототип, который осуществляет генерацию байткода для двух подоходов.

\subsubsection{Реализация с помощью рефлексии}
Для каждого места вызова генерируется статический метод. Этот метод принимает на вход объект на котором происходит вызов и возращает экземпляр класса Method. После этого все аргументы упаковываются в массив и вызывается метод invoke, который делает вызов нужного метода и сохраняет на стек результат. Поскольку сигнатура метода invoke не типизирована, то необходимо выполнить приведение типов.

\subsubsection{Реализация с помощью invokedynamic}
Рассмотрим реализацию с помощью invokedynamic. Аналогично предыдущему варианту, для каждого места вызова генерируется статический метод. В этом вызове происходит вызов инструкции invokedynamic, которой передаётся ссылка на вспомогательный статический метод и тип вызываемого метода. Помимо этого JVM передаёт экземпляр класса Lookup для данного места вызова. Внутри вспомогательного метода создаётся объект типа ConstantCallSite который содержит в себе вспомогательный класс, содержащий методы отвечающие за поиск нужного метода. Для поиска метода происходит вызов метода find для объекта, который возвращает экземпляр класса MethodHandle, ссылающийся на необходимый метод. Из вызова статического метода возвращается MethodHandle, вызов у него invokeExact вызывает необходимый метод. В отличии от механизма рефлексии, оборачивание аргументов в массив не требуется.

\subsubsection{Сравнение производительности}
Для реализации было предположено что типов, на которых делается вызов(в одном месте вызова), будет существенно меньше чем самих вызовов. Поиск метода выполняется не на самом объекте, а на его типе. Поэтому для каждого типа, на котором происходит вызов, храним найденный метод. Это позволит сильно сохранить время вызова.

\paragraph{Условия тестирования}
Тестирование проводилось на следующей конфигурации компьютера:
\begin{itemize}
    \item CPU Intel i7 6700 3.4 Ghz
    \item RAM 32GB DDR4
    \item Oracle JDK 1.8
    \item OS Ubuntu desktop 17.04 x64
    \item JMH 1.16
\end{itemize}

Цель данного измерения понять скорость вызова метода с помощью двух решений в различных случаях. Тестирование производилось для функции с одним аргументов, для одного типа на один тест. Вызовы производились в одном потоке. Внутри которой вызывается специальный метод JMH предотвращающий удаление непродуктивного кода. В процессе тестирования было обнаружено, что существует влияние типа реализации протокола на скорость вызова. Время выполнения измерялось в наносекундах на вызов, каждый бенчмарк запускался в течении секунды, результаты были усреднены по 20 запускам. Результаты измерений для рефлексии приведены в таблице \ref{benchmark:prototype:reflection}.
\begin{table}[h]
\begin{center}
\begin{tabular}{|c|c|} \hline
Название бенчмарка & Время работы (нс) \\ \hline
Вызов метода класса & 4.815 $\pm$ 0.072 \\ \hline
Вызов метода наследника через ссылку на базовый класс & 4.797 $\pm$ 0.023 \\ \hline
Вызов метода интерфейса & 4.818 $\pm$ 0.055 \\ \hline
Вызов метода интерфейса реализованного анонимной функцией & 153.892 $\pm$ 0.909 \\ \hline
\end{tabular}
\caption{Результаты бенчмарка для реализации с помощью рефлексии}
\label{benchmark:prototype:reflection}
\end{center}
\end{table}

Из измерений видно что вызов метода реализованного анонимной функцией в 30 раз медленнее вызова обычного метода. Аналогичные измерения для библиотеки invoke приведены в таблице \ref{benchmark:prototype:invoke}.

\begin{table}[h]
\begin{center}
\begin{tabular}{|c|c|} \hline
Название бенчмарка & Время работы (нс) \\ \hline
Вызов метода класса & 5.002 $\pm$ 0.031 \\ \hline
Вызов метода наследника через ссылку на базовый класс & 5.002 $\pm$ 0.037 \\ \hline
Вызов метода интерфейса & 5.016 $\pm$ 0.104 \\ \hline
Вызов метода интерфейса реализованного анонимной функцией & 5.069 $\pm$ 0.111 \\ \hline
\end{tabular}
\caption{Результаты бенчмарка для реализации с помощью библиотеки invoke}
\label{benchmark:prototype:invoke}
\end{center}
\end{table}

В реализации с помощью invoke все случаи использования работают с одной скоростью не зависимо от целевого объекта. В двух реализациях скорость является схожей, поэтому для дальнейшего изучения было принято решение реализовать оба способа в компиляторе и провести дальнейшие тестирования.

\subsection{Реализация протоколов в компиляторе Kotlin}

Поддержка протоколов в компиляторе заключается в введении способа объявить тип протокола и определении набора правил для приведения и вызове метода. Для того, чтобы объявить тип протокола в компиляторе был введён новый тип интерфейса. Если перед именем интерфейса указать ключевое слово \term{porotocol}, интерфейс будет автоматически считаться объявлением протокола типа:

\begin{minted}{kotlin}
protocol interface Proto {
  fun id(i: Int): Int
}

class Impl {
  fun id(i: Int): Int {
    return i
  }
}

fun foo(arg: Proto) {
  println(arg.id(42))
}

foo(Impl())
\end{minted}

Объявлённый протокол может использоваться так же, как и обычный интерфейс: являться типовым параметром, быть родительским классом и т.д. Однако информация о нём отсутствует во время выполнения: у классов, явно реализующих тип протокола, он отсутствует в списке интерфейсов.

Для того чтобы избежать неоднозначного поведения, в протоколах отключены методы и аргументы по умолчанию.

\subsubsection{Выбор метода}

В зависимости от способов объявления и реализации протокола, у разных типов приведённых к протоколу, могут быть разные методы в одном месте вызова. Помимо этого, в языке \term{Kotlin} поддерживаются перегрузки методов с разным числом и типом параметров. Подобные ситуации могут приводить к неоднозначности в выборе метода.

Примером такой ситуации может послужить использование параметрических типов в протоколе:
\begin{minted}{kotlin}
protocol interface Sample<T> {
  fun foo(arg: T)
}

class X {
  fun foo(i: Int)
  fun foo(i: Int?)
}

...
val x = X()
val first: Sample<Int> = x
val second: Sample<Int?> = x

\end{minted}

На уровне виртуальной машины отсутствует информация о типовых параметрах. В случае метода \term{foo(Int)} транслированный тип будет примитивным, а в \term{foo(Int?)} ссылочным. Если бы тип \term{Sample} не был протоколом, то из двух ссылок \term{first} и \term{second} был бы доступен метод \term{foo(Int?)}. В случае протокола, будет запущен алгоритм выбора метода, который выберет наиболее близкий по типам метод в каждом случае. В данном случае, из ссылки \term{first} доступен для вызова метод \term{foo(Int)}, а из ссылки \term{second} доступен \term{foo(Int?)}.

Алгоритм выбора метода независимо запускается во время компиляции и во время выполнения. Выбор метода во время компиляции необходим при приведении типа объекта к типу протокола. В данном случае, происходит проверка наличия всех методов протокола в объекте во время компиляции. В случае, если не существует метода с необходимым возвращаемым значением, но существует метод с более конкретным типом возвращаемого значения, то он будет считаться подходящим.

Во время выполнения, выбор метода происходит непосредственно во время вызова. Так как представление компилятора и виртуальной машины о методах различны, сохранить результат выбора во время компиляции как есть не представляется возможным. Данный метод гарантировано существует, так как это было проверено на этапе компиляции.

\subsubsection{Исключения}
Вызов метода протокола отличается от вызова обычного метода. Некоторые реализации могут предполагать оборачивание исключения в вспомогательный класс. В случае, если в методе протокола происходит исключение, оно будет передано без оборачивания, независимо от типа реализации протокола.

\subsubsection{Типовые параметры}
В отличии от реализаций протоколов для других \tool{JVM} языков, протоколы в \lang{Kotlin} полностью поддерживают работу с типовыми параметрами, так же как и обычные типы. Протокол может являться, как типовым аргументом, так и самостоятельно содержать типовые параметры. Более того, реализация параметризованного класса может являться подтипом непараметризованного протокола, и наоборот: реализация непараметризованного класса может быть подтипом параметризованного протокола с подставленным типом.

\paragraph{Операторы проверки и приведения типа}
Для обычных типов в \lang{Kotlin} разрешено приведение или проверка любого объекта к любому типу. В большинстве случаев, объекты, реализующие протоколы, не являются типами протоколов, с точки зрения \tool{JVM}. В таком случае возможно два поведения: реальная проверка типов инструкцией \tool{JVM}, либо проверка соответствя протоколу во время выполнения. Оба способа возможно реализовать практически. С точки зрения пользователя, каждое из поведений может являться не ожидаемым, поэтому было принято решение запретить использование протоколов в правой части операторов \operator{as} и \operator{is}. В тоже время, разрешено присваивание объекта обычного типа к типу протокола, если проверка ограничений на тип проходит во время компиляции. Если извесен исходный тип объекта типа протокола, то можно выполнить приведение типов с помощью встроенных операторов. Так же можно выполнить проверку типов на объекте типа протокола. Таким образом существует возможность перехода от обычных типов к протоколам и от протоколов к обычным типам.

