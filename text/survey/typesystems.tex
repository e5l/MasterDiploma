\subsection{Структурная и номинальная типизации}
В данном разделе рассмотрены основы структурной типизации и номинальной типизации.

Номинальная типизация во многих языках реализованна с помощью механизма наследования. Пользователь явно указывает наследников каждого класса. Исходя из этого, компилятор разрешает вызовы методов. В некоторых языках происходят опциональные проверки перегрузок.

Плюсы номинальной типизации:
\begin{itemize}
    \item быстрее во время выполнения
    \item помогает избежать случайных отношений
    \item проще объявлять рекурсивные типы
    \item быстрее проверять
\end{itemize}

Структурная типизация является более гибкой. Тип $A$ является подтипом типа $B$ если выполнены следующие условия:
\begin{itemize}
    \item Тип $A$ и тип $B$ одинаковое количество полей
    \item $A$ содержит все именованные поля $B$
    \item Тип каждого поля $A$ является подтипом $B$
\end{itemize}
Типы функций являются подтипами только при полном совпадении типов аргументов и возвращаемого значения. Имена типов при этом не имеют значения: два типа с одинаковым содержимым. Проверка типов при структурной типизации: проверка вложенности двух типов.

Плюсы структурной типизации:
\begin{itemize}
    \item замкнуты: тип содержит в себе всё описание
    \item имя типа имеет символический характер и не влияет на отношение типизации
\end{itemize}